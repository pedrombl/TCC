\section{Conclusions}

In this monograph, we first defined the scalability, and a function that quantifies it. To analyze the scalability of an application, using the function defined earlier, a set of steps was described. The first version of the Scalability Testing Framework was developed to support the scalability explorer to analyze the scalability of an application. It automates some of these steps, and assists the visualization of the results by plotting a graph with them.

We have done two validations with it, the Distributed Matrix Multiplication and the Supermarket of the Future Choreography. In the first one, we compared two different approaches with different scalabilities. Using the framework, we saw the difference between them in the graph, and evaluated which approach was more scalable. Therefore, the framework proved to be a good tool for comparing two similar implementations, and verifying the impact of this difference in the scalability.

In the second one, we analyzed the scalability of only one approach. Consequently, we could not compare to other approaches, and evaluate which one was more scalable. We concluded that to analyze the scalability with only one implementation another steps should be done, such as defining an upper bound for the performance metric, and calculating the derived of the curve from the graph to analyze if this value is too high or not. If we have made other changes in the implementation, we would have two slightly different architectures to compare the scalability. 

Since the scalability explorer defines the procedures of scalability testing in the framework specifying test methods, the application can pass through different kinds of scalability assessments. This is important on applications that have different variables that can affect its scalability, such as choreographies that are composed of different layers of implementation.

\subsection{On Going Work}

A first experiment on choreographies has been done. However, we need to continue making experiments with them to explore the different scalability types that are important to choreographies. 

In addition, we should develop tools that manipulate cloud instances, where the choreography is usually deployed, to increase the architecture capability. This will make easier the development of scalability tests for choreographies. CHOReOS project is developing a series of tools to manipulate them, such as the Node Pool Manager\footnote{https://github.com/choreos/}, which manages cloud instances. Nevertheless, these tools might need to be adapted to suit the Scalability Testing Framework.

In this first version, the framework does not make a conclusion of how much the choreography is scalable by itself. Therefore, we intend to develop a heuristic to determine the scalability size of an application, based on a deeply scalability research, and on the experiments with choreographies.